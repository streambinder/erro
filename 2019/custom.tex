\documentclass{article}

\usepackage[nodisplayskipstretch]{setspace}
\usepackage[sfdefault,condensed]{roboto}
\usepackage[utf8]{inputenc}
\usepackage[T1]{fontenc}
\usepackage{array}
\usepackage{color}
\usepackage{fancyhdr}
\usepackage{fontawesome}
\usepackage{geometry}
\usepackage{graphicx}
\usepackage{hyperref}

\urlstyle{same}

\definecolor{deepblue}{rgb}{.1,.2,.3}
\definecolor{lightgray}{rgb}{.45,.45,.45}
\definecolor{lightergray}{rgb}{.75,.75,.75}

\newcolumntype{L}[1]{>{\raggedright\let\newline\\\arraybackslash\hspace{0pt}}m{#1}}
\newcolumntype{C}[1]{>{\centering\let\newline\\\arraybackslash\hspace{0pt}}m{#1}}
\newcolumntype{R}[1]{>{\raggedleft\let\newline\\\arraybackslash\hspace{0pt}}m{#1}}

\pagestyle{fancy}
\fancyhf{}
\renewcommand{\headrulewidth}{0pt}
\fancyfoot[C]{
	\tiny{\color{lightgray}{{{year}}}} \\
	\scriptsize{{{footnote}}}
}

\begin{document}
\newgeometry{margin=3cm}

% header
\begin{minipage}[t]{.25\textwidth}
	\raisebox
	{\dimexpr-\height+\ht\strutbox\relax}
	{\includegraphics[width=1\textwidth]{profile.png}} \hfill \\\\\\\\
\end{minipage}
\begin{minipage}[t]{.7\textwidth}
	\begin{flushright}
		\vfill \textbf{\huge{Davide Pucci}} \\ \hfill \vfill
		\textit{\setstretch{1.15}
			Via Rocca Sinibalda 10, 00199, Roma (RM), Italia \hspace{1mm} \color{lightergray}\faHome\color{black} \\
			+39 3298580201                                   \hspace{1mm} \color{lightergray}\faPhone\color{black} \\
			posta@davidepucci.it                             \hspace{1mm} \color{lightergray}\faEnvelope\color{black} \\
			21/09/1995                                       \hspace{1mm} \color{lightergray}\faCalendar\color{black} \\
			\url{https://github.com/streambinder}            \hspace{1mm} \color{lightergray}\faGithubSquare\color{black} \\
			\url{https://linkedin.com/in/davide-pucci}       \hspace{1mm} \color{lightergray}\faLinkedinSquare\color{black} \\
		}
	\end{flushright}
\end{minipage}

% paragraph title
\textbf{\textcolor{deepblue}{\uppercase{{{experience}}}}} \\\\ \hfill
% paragraph content
\begin{tabular}{ R{3cm} l }
	\textcolor{lightgray}{12/2018 — {{ongoing}}} & \begin{tabular}[l]{@{}l@{}}
		\textbf{Consulente Software Engineering}, presso \textit{Kubique S.p.A.} \\
		La collaborazione tra \textit{I-Node S.r.l.} e \textit{Kubique S.p.A.} è stata \\
		l'occasione attraverso cui poter affacciarmi nel mondo dello sviluppo di \textit{Java EE}, \\
		poter imparare a gestire requisiti complessi come la necessità di rendere un prodotto \\
		tanto elastico quanto stabile ed entrare a contatto con nuovi paradigmi di sviluppo, \\
		come la \textit{programmazione orientata agli aspetti}.
	\end{tabular} \\\\ \hfill
	\textcolor{lightgray}{10/2015 — 12/2018} & \begin{tabular}[l]{@{}l@{}}
		\textbf{Software / DevOps Engineer}, presso \textit{I-Node S.r.l.} \\
		Nell'approcciare alle metodologia \textit{DevOps}, ho imparato a realizzare \\
		e/o configurare servizi sufficientemente elastici e solidi da garantirne i deployments \\
		automatici e periodici, che non compromettessero in alcun modo la struttura realizzata. \\
		L'avvicinamento a questo mondo ha imposto un metodo di programmazione \\
		che non fosse cieco e interessato soltanto al prodotto in fase di realizzazione, \\
		ma che tenesse conto dell'ecosistema che gli avrebbe fatto da contenitore.
	\end{tabular} \\\\ \hfill
	\textcolor{lightgray}{07/2014 — {{ongoing}}}   & \begin{tabular}[l]{@{}l@{}}
		\textbf{Systems Engineer}, presso \textit{I-Node S.r.l.} \\
		Ho avuto modo di entrare a contatto con problematiche complesse che mi hanno \\
		portato ad una crescita tale da essere in grado di gestire problemi tecnici \\
		interfacciandomi con clienti e/o supporto tecnico di terze parti. \\
		L'eterogeneità del tipo di attivit\`a mi ha permesso di coprire una vasta gamma \\
		di ambiti, dal semplice supporto Tier 1 / Tier 2 alla realizzazione di infrastrutture \\
		informatiche complesse, dal deploy di web/mail server, all'irrobustimento dei \\
		sistemi firewall, dalla configurazione di sistemi di monitoraggio, alla messa in \\
		sicurezza di sistemi, e molto altro.
		\end{tabular} \\\\ \hfill
\end{tabular}

% paragraph title
\textbf{\textcolor{deepblue}{\uppercase{{{education}}}}} \\\\ \hfill
% paragraph content
\begin{tabular}{ R{3cm} l }
	\textcolor{lightgray}{2014 — 2019} & \begin{tabular}[l]{@{}l@{}}
	\textbf{Laurea in Informatica} \\
	Presso: \textit{La Sapienza, Universit\`a di Roma} \\
\end{tabular} \\\\ \hfill
\textcolor{lightgray}{2009 — 2014}  & \begin{tabular}[l]{@{}l@{}}
\textbf{Diploma di liceo scientifico} \\
Presso: \textit{Liceo Scientifico Statale Amedeo Avogadro} \\
Punteggio: 76/100
\end{tabular} \\\\ \hfill
\end{tabular}

\newpage

% paragraph title
\textbf{\textcolor{deepblue}{\uppercase{{{projects}}}}} \\\\ \hfill
% paragraph content
\begin{tabular}{ R{3cm} l }
	\textcolor{lightgray}{2017 — {{ongoing}}} &
		\begin{tabular}[l]{@{}l@{}}
			\textbf{Spaghetti DevOps}, \textit{Davide Pucci}, \textit{Marco Panunzio}, \textit{Fabrizio Pietrucci}. \\
			Insieme a due colleghi universitari ho intrapreso un'attivit\`a di blogging che \\
			descriva problemi - ed eventuali soluzioni ad essi - in cui \`e possibile imbattersi \\
			lavorando nel panorama informatico. \\
			Nasce principalmente per hobby, ma \`e spinto dall'interesse comune di realizzare \\
			una knowledge base per il generico lavoratore nel campo IT.
		\end{tabular} \\\\ \hfill
	\textcolor{lightgray}{2012 — 2014} &
		\begin{tabular}[l]{@{}l@{}} %
			\textbf{MoltenMotherBoard}, \textit{Davide Pucci}, \textit{Marco Panunzio}, \textit{Giovanni Santini}. \\
			Per due anni ho sviluppato nel mondo Android sotto il nome di MoltenMotherBoard,                       \\
			progetto iniziato con altri due colleghi. Con questo team sono stati rilasciati                        \\
			custom ROMs, kernel e recovery, per diversi terminali, quali il                                        \\
			Samsung Galaxy Gio GT-S5660, il Samsung Galaxy Pocket GT-S5300,                                        \\
			il Samsung Galaxy Ace Plus GT-S7500 e l'LG Optimus 4X HD.
		\end{tabular} \\\\ \hfill
\end{tabular}

% paragraph title
\textbf{\textcolor{deepblue}{\uppercase{{{skills_networking}}}}} \\\\ \hfill
% paragraph content
\begin{tabular}{ R{4cm} l }
	\textcolor{lightgray}{Hosting web}             & Apache, Nginx, ISPConfig, Plesk                              \\ \hfill
	\textcolor{lightgray}{Hosting email}           & Postfix, Courier, Dovecot, Zimbra, Roundcube                 \\ \hfill
	\textcolor{lightgray}{Hosting file}            & VSFTP, SFTP                                                  \\ \hfill
	\textcolor{lightgray}{DBMS}                    & MySQL / MariaDB, PostgreSQL, LightDB, OrientDB               \\ \hfill
	\textcolor{lightgray}{Monitoraggio}            & Nagios, Icinga(2), NagiosQL, Cacti, Netdata                  \\ \hfill
	\textcolor{lightgray}{Backup}                  & Bacula, Veeam                                                \\ \hfill
	\textcolor{lightgray}{Rete}                    & IPTables, IPsec, OpenVPN, GateProtect                        \\ \hfill
	\textcolor{lightgray}{Controllo remoto}        & Samba(4), NFSv4, Kerberos                                    \\ \hfill
	\textcolor{lightgray}{Automazione di sviluppo} & Make, Maven, Gradle                                          \\ \hfill
	\textcolor{lightgray}{Virtualizzazione}        & VMware vSphere, VirtualBox, Vagrant, Docker                  \\ \hfill
	\textcolor{lightgray}{Controllo versione}      & Git, SVN                                                     \\ \hfill
	\textcolor{lightgray}{Piattaforme cloud}       & OwnCloud, NextCloud                                          \\ \hfill
	\textcolor{lightgray}{Antivirus}               & Sophos Enterprise Console, Sophos Endpoint Antivirus, G Data \\ \hfill
	\textcolor{lightgray}{Sistemi Operativi}       & Linux, Android, Windows, Windows Server                      \\ \hfill
\end{tabular}

% paragraph title
\textbf{\textcolor{deepblue}{\uppercase{{{skills_development}}}}} \\\\ \hfill
% paragraph content
\begin{tabular}{ R{4cm} l }
	\textcolor{lightgray}{Base}       & Ruby, MySQL                                    \\ \hfill
	\textcolor{lightgray}{Intermedio} & C, PHP, LaTeX, JavaScript/TypeScript, ASM MIPS \\ \hfill
	\textcolor{lightgray}{Avanzato}   & Java, Python, Bash, GoLang                     \\[.5cm] \hfill
	\textcolor{lightgray}{Frameworks} & Laravel, Spring Boot, JSF/Hibernate, Ionic     \\ \hfill
\end{tabular}

% paragraph title
\textbf{\textcolor{deepblue}{\uppercase{{{languages}}}}} \\\\ \hfill
% paragraph content
\begin{tabular}{ R{4cm} l }
	\textcolor{lightgray}{Italiano} & Madrelingua \\ \hfill
	\textcolor{lightgray}{Inglese}  & B1          \\ \hfill
\end{tabular}

\end{document}
