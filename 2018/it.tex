\documentclass{article}

\usepackage[nodisplayskipstretch]{setspace}
\usepackage[sfdefault,condensed]{roboto}
\usepackage[utf8]{inputenc}
\usepackage[T1]{fontenc}
\usepackage{array}
\usepackage{color}
\usepackage{fancyhdr}
\usepackage{fontawesome}
\usepackage{geometry}
\usepackage{graphicx}
\usepackage{hyperref}

\urlstyle{same}

\definecolor{deepblue}{rgb}{.1,.2,.3}
\definecolor{lightgray}{rgb}{.45,.45,.45}
\definecolor{lightergray}{rgb}{.75,.75,.75}

\newcolumntype{L}[1]{>{\raggedright\let\newline\\\arraybackslash\hspace{0pt}}m{#1}}
\newcolumntype{C}[1]{>{\centering\let\newline\\\arraybackslash\hspace{0pt}}m{#1}}
\newcolumntype{R}[1]{>{\raggedleft\let\newline\\\arraybackslash\hspace{0pt}}m{#1}}

\pagestyle{fancy}
\fancyhf{}
\renewcommand{\headrulewidth}{0pt}
\fancyfoot[C]{
	\tiny{\color{lightgray}{2018}} \\
	\scriptsize{Autorizzo il trattamento dei miei dati personali ai sensi della legge sulla privacy 196/03 e successive modifiche e integrazioni.}
}

\begin{document}
\newgeometry{margin=3cm}

% header
\begin{minipage}[t]{.25\textwidth}
	\raisebox
	{\dimexpr-\height+\ht\strutbox\relax}
	{\includegraphics[width=1\textwidth]{profile.png}} \hfill \\\\\\\\
\end{minipage}
\begin{minipage}[t]{.7\textwidth}
	\begin{flushright}
		\vfill \textbf{\huge{Davide Pucci}} \\ \hfill \vfill
		\textit{\setstretch{1.15}
			Via Rocca Sinibalda 10, 00199, Roma (RM), Italia \hspace{1mm} \color{lightergray}\faHome\color{black} \\
			+39 3298580201                                   \hspace{1mm} \color{lightergray}\faPhone\color{black} \\
			posta@davidepucci.it                             \hspace{1mm} \color{lightergray}\faEnvelope\color{black} \\
			21/09/1995                                       \hspace{1mm} \color{lightergray}\faCalendar\color{black} \\
			\url{https://github.com/streambinder}            \hspace{1mm} \color{lightergray}\faGithubSquare\color{black} \\
			\url{https://linkedin.com/in/davide-pucci}       \hspace{1mm} \color{lightergray}\faLinkedinSquare\color{black} \\
		}
	\end{flushright}
\end{minipage}

% paragraph title
\textbf{\textcolor{deepblue}{ESPERIENZA}} \\\\ \hfill
% paragraph content
\begin{tabular}{ R{3cm} l }
	\textcolor{lightgray}{10/2015 — in corso} & \begin{tabular}[l]{@{}l@{}}
	\textbf{Software / DevOps Engineer}, presso \textit{I-Node S.r.l.} \\
	L'avvio di alcuni progetti web ha reso necessaria l'applicazione della metodologia \\
	DevOps. Nell'approcciare ad essa, ho imparato a realizzare e/o configurare servizi \\
	sufficientemente elastici da garantire deployments automatici e periodici, che non \\
	compromettessero in alcun modo la struttura realizzata e a consolidare le mie \\
	capacità di sviluppatore. Infatti, l'avvicinamento a questo mondo ha imposto un \\
	metodo di programmazione che non fosse cieco e interessato soltanto al prodotto \\
	in fase di realizzazione, ma che tenesse conto dell'ecosistema che gli avrebbe \\
	fatto da contenitore. \\
	Nella fattispecie, ho preso dimestichezza con tecnologie quali Jboss / Wildfly, \\
	il pattern architetturale MVC, i software di build automation Maven e Gradle.
\end{tabular} \\\\ \hfill
\textcolor{lightgray}{07/2014 — in corso}   & \begin{tabular}[l]{@{}l@{}}
\textbf{Systems Engineer}, presso \textit{I-Node S.r.l.} \\
Ho avuto modo di entrare a contatto con problematiche complesse che mi hanno \\
portato ad una crescita tale da essere in grado di gestire problemi tecnici \\
interfacciandomi con clienti e/o supporto tecnico di terze parti. \\
L'eterogeneità del tipo di attivit\`a mi ha permesso di coprire una vasta gamma \\
di ambiti, dal semplice supporto Tier 1 / Tier 2 alla realizzazione di infrastrutture \\
informatiche complesse, dal deploy di web/mail server, all'irrobustimento dei \\
sistemi firewall, dalla configurazione di sistemi di monitoraggio, alla messa in \\
sicurezza di sistemi, e molto altro.
\end{tabular} \\\\ \hfill
\end{tabular}

% paragraph title
\textbf{\textcolor{deepblue}{EDUCAZIONE}} \\\\ \hfill
% paragraph content
\begin{tabular}{ R{3cm} l }
	\textcolor{lightgray}{2014 — in corso} & \begin{tabular}[l]{@{}l@{}}
	\textbf{Laurea in Informatica} \\
	Presso: \textit{La Sapienza, Universit\`a di Roma} \\
\end{tabular} \\\\ \hfill
\textcolor{lightgray}{2009 — 2014}  & \begin{tabular}[l]{@{}l@{}}
\textbf{Diploma di liceo scientifico} \\
Presso: \textit{Liceo Scientifico Statale Amedeo Avogadro} \\
Punteggio: 76/100
\end{tabular} \\\\ \hfill
\end{tabular}

\newpage

% paragraph title
\textbf{\textcolor{deepblue}{PROGETTI}} \\\\ \hfill
% paragraph content
\begin{tabular}{ R{3cm} l }
	\textcolor{lightgray}{2017 — in corso} &
	\begin{tabular}[l]{@{}l@{}} %
	\textbf{Spaghetti DevOps}, \textit{Davide Pucci}, \textit{Marco Panunzio}, \textit{Fabrizio Pietrucci}. \\
	Insieme a due colleghi universitari ho intrapreso un'attivit\`a di blogging che \\
	descriva problemi - ed eventuali soluzioni ad essi - in cui \`e possibile imbattersi \\
	lavorando nel panorama informatico. \\
	Nasce principalmente per hobby, ma \`e spinto dall'interesse comune di realizzare \\
	una knowledge base per il generico \textit{IT-addict}.
\end{tabular} \\\\ \hfill
\textcolor{lightgray}{2012 — 2014} &
\begin{tabular}[l]{@{}l@{}} %
	\textbf{MoltenMotherBoard}, \textit{Davide Pucci}, \textit{Marco Panunzio}, \textit{Giovanni Santini}. \\
	Per due anni ho sviluppato nel mondo Android sotto il nome di MoltenMotherBoard,                       \\
	progetto iniziato con altri due colleghi. Con questo team sono stati rilasciati                        \\
	custom ROMs, kernel e recovery, per diversi terminali, quali il                                        \\
	Samsung Galaxy Gio GT-S5660, il Samsung Galaxy Pocket GT-S5300,                                        \\
	il Samsung Galaxy Ace Plus GT-S7500 e l'LG Optimus 4X HD.
\end{tabular} \\\\ \hfill
\end{tabular}

% paragraph title
\textbf{\textcolor{deepblue}{ABILIT\`A IN RETE/SISTEMI}} \\\\ \hfill
% paragraph content
\begin{tabular}{ R{4cm} l }
	\textcolor{lightgray}{Web hosting}       & Apache, Nginx, ISPConfig, Plesk                              \\ \hfill
	\textcolor{lightgray}{Mail hosting}      & Postfix, Courier, Dovecot, Zimbra, Roundcube                 \\ \hfill
	\textcolor{lightgray}{File hosting}      & VSFTP, SFTP                                                  \\ \hfill
	\textcolor{lightgray}{DBMS}              & MySQL / MariaDB, PostgreSQL, LightDB, OrientDB               \\ \hfill
	\textcolor{lightgray}{Monitoring}        & Nagios, Icinga, NagiosQL, Cacti, Netdata                     \\ \hfill
	\textcolor{lightgray}{Backup}            & Bacula, Veeam                                                \\ \hfill
	\textcolor{lightgray}{Networking}        & IPTables, IPsec, OpenVPN, GateProtect                        \\ \hfill
	\textcolor{lightgray}{SW Deployment}     & OCS Inventory                                                \\ \hfill
	\textcolor{lightgray}{Virtualization}    & VMware vSphere, VirtualBox, Vagrant, Docker                  \\ \hfill
	\textcolor{lightgray}{Version Control}   & Git, SVN                                                     \\ \hfill
	\textcolor{lightgray}{Piattaforme cloud} & OwnCloud, NextCloud                                          \\ \hfill
	\textcolor{lightgray}{Antivirus}         & Sophos Enterprise Console, Sophos Endpoint Antivirus, G Data \\ \hfill
	\textcolor{lightgray}{Operating Systems} & Linux, Android, Windows, Windows Server                      \\ \hfill
\end{tabular}

% paragraph title
\textbf{\textcolor{deepblue}{ABILIT\`A IN SVILUPPO}} \\\\ \hfill
% paragraph content
\begin{tabular}{ R{4cm} l }
	\textcolor{lightgray}{Base}       & Ruby, JavaScript, MySQL       \\ \hfill
	\textcolor{lightgray}{Intermedio} & C, PHP, LaTeX                 \\ \hfill
	\textcolor{lightgray}{Avanzato}   & Java, Python, Bash, GoLang    \\\\ \hfill
	\textcolor{lightgray}{Frameworks} & Laravel, JSF/Hibernate, Ionic \\ \hfill
\end{tabular}

% paragraph title
\textbf{\textcolor{deepblue}{LINGUE}} \\\\ \hfill
% paragraph content
\begin{tabular}{ R{4cm} l }
	\textcolor{lightgray}{Italiano} & Madrelingua \\ \hfill
	\textcolor{lightgray}{Inglese}  & B1          \\ \hfill
\end{tabular}

\end{document}
